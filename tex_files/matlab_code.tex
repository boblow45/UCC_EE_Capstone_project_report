	
	\chapter{MatLab Code}
	
	\definecolor{codegreen}{rgb}{0,0.6,0}
	\definecolor{codegray}{rgb}{0.5,0.5,0.5}
	\definecolor{codepurple}{rgb}{0.58,0,0.82}
	\definecolor{backcolour}{rgb}{0.95,0.95,0.92}
	
	\lstdefinestyle{mystyle}{
		backgroundcolor=\color{backcolour},   
		commentstyle=\color{codegreen},
		keywordstyle=\color{magenta},
		numberstyle=\tiny\color{codegray},
		stringstyle=\color{codepurple},
		%basicstyle=\footnotesize  ,
		basicstyle=\scriptsize ,
		breakatwhitespace=false,         
		breaklines=true,                 
		captionpos=b,                    
		keepspaces=true,                 
		numbers=left,                    
		numbersep=5pt,                  
		showspaces=false,                
		showstringspaces=false,
		showtabs=false,                  
		tabsize=2,
		language=Matlab,
		rulecolor=\color{black},
		xleftmargin=\parindent,
		escapeinside={\%*}{*)},          % if you want to add LaTeX within your code
		morekeywords={tonyplot,line,loc,spac,diffuse,resistivity,orientation,init}
	}
	
	\lstset{style=mystyle}
	
	\section{Complementary Filter (MatLab Code)}
   \subsection{First Order Complementary Filter} \label{sec: first order complmentary filter code}
    \begin{lstlisting} [caption=  MatLab Code Used to Implement a First Order Complementary Filter, label=lst: first order comp filter]
   function [pitch_est_comp,roll_est_comp] =comp_filter_first_tustin(acc_data,gyro_data,ts,tau_roll,tau_pitch) 
   % First Order Complamentry Filter used to estimate pitch and roll of the quad-rotor
   % function [pitch_est_comp,roll_est_comp] =comp_filter_first_tustin(acc_data,gyro_data,ts,tau_roll,tau_pitch) 
   %
   % Inputs:-
   %	acc_data    : A logged form of the accelerometer data (denoted $a_x$, $a_y$ and $a_z$ in this report, also this input has to be in an array formate)
   % 	gyro_data   : A logged form of the gyroscope data (denoted $w_x$, $w_y$ and $w_z$ in this report, also this input has to be in an array formate)
   %	ts          : Sampleing of the controller used to collected the gyroscope and accelerometer data
   %	tau_roll    : Filtering coeffient for the roll axis (sets the cutoff frequency of the filter)
   %	tau_pitch   : Filtering coeffient for the pitch axis (sets the cutoff frequency of the filter)
   %
   % Output:-  
   % 	pitch_est_comp : Outputs the pitch estimate for the given cutoff frequency (in an array formate)
   % 	roll_est_comp  : Outputs the roll estimate for the given cutoff frequency  (in an array formate)
   %
   %Note this Complementary Filter code used Tustin's Methodd to emulate the analog filter to the discrete domain.
   
   wx = [0;gyro_data(:,1)];
   wy = [0;gyro_data(:,2)];
   wz = [0;gyro_data(:,3)];
   
   %model of the filter
   
   pitch = atan(-acc_data(:,1)./(sqrt(acc_data(:,2).^2 + acc_data(:,3).^2))); 
   roll  = atan(acc_data(:,2)./acc_data(:,3));    
   
   pitch = [0,pitch'];
   roll =  [0,roll'];
   
   pitch_est_comp=[0.0];
   roll_est_comp=[0.0];
   
   alpha_roll  = ts - 2*tau_roll;
   beta_roll   = ts + 2*tau_roll;
   
   alpha_pitch = ts - 2*tau_pitch;
   beta_pitch  = ts + 2*tau_pitch;
   
   for  i=1:(length(acc_data(:,1))-1)
   
   roll_est_comp(i+1)  = 1/beta_roll*(  ts*(roll(i+1) + roll(i)) -alpha_roll*roll_est_comp(i) + tau_roll*ts*(wx(i+1) + wx(i)));
   pitch_est_comp(i+1) = 1/beta_pitch*(  ts*(pitch(i+1) + pitch(i)) -alpha_pitch*pitch_est_comp(i) + tau_pitch*ts*(wy(i+1) + wy(i)));
   end
   \end{lstlisting}
   
   
   
      \subsection{Second Order Complementary Filter} \label{sec: second order complmentary filter code}
      \begin{lstlisting} [caption=  MatLab Code Used to Implement a Second Order Complementary Filter, label=lst: second order comp filter]
     function [pitch_est_comp,roll_est_comp]=comp_filter_second_order(acc_data,gyro_data,ts,roll_parameters,pitch_parameters)  
     % Second Order Complamentry Filter used to estimate pitch and roll of the quad-rotor
     % function [pitch_est_comp,roll_est_comp]=comp_filter_second_order(acc_data,gyro_data,ts,roll_parameters,pitch_parameters) 
     %
     %
     % Inputs:_
     %           acc_data         : A logged form of the accelerometer data (denoted $a_x$, $a_y$ and $a_z$ in this report, also this input has to be in an array formate)
     %           gyro_data        : A logged form of the gyroscope data (denoted $w_x$, $w_y$ and $w_z$ in this report, also this input has to be in an array formate)
     %           ts               : Sampleing of the controller used to collected the gyroscope and accelerometer data
     %           roll_parameters  : Array containing the roll tuning parameters as follows [damping, wo]
     %           pitch_parameters : Array containing the pitch tuning parameters as follows [damping, wo]
     %
     % Output:-  
     %          pitch_est_comp : Outputs the pitch estimate for the given cutoff frequency (in an array formate)
     %          roll_est_comp  : Outputs the roll estimate for the given cutoff frequency  (in an array formate)
     
     ki_roll      =roll_parameters(2)^2;
     kp_roll      =2*roll_parameters(1)*roll_parameters(2);
     
     ki_pit      =    pitch_parameters(2)^2; 
     kp_pit      =  2*pitch_parameters(1)*pitch_parameters(2);  
     
     %varible for the disscert filter
     Gamma_pit = ki_pit*ts^2 - 2*kp_pit*ts;
     alpha_pit = Gamma_pit + 4;
     xi_pit    = 2*ki_pit*ts^2;
     eta_pit   = ki_pit*ts^2 + 2*kp_pit*ts;
     beta_pit  = xi_pit - 8;
     delta_pit = eta_pit + 4;
     
     Gamma_roll = ki_roll*ts^2 - 2*kp_roll*ts;
     alpha_roll = Gamma_roll + 4;
     xi_roll    = 2*ki_roll*ts^2;
     eta_roll   = ki_roll*ts^2 + 2*kp_roll*ts;
     beta_roll  = xi_roll - 8;
     delta_roll = eta_roll + 4;
     
     sigma = 2*ts;
     
     wx = [0;0;gyro_data(:,1)];
     wy = [0;0;gyro_data(:,2)];
     wz = [0;0;gyro_data(:,3)];
     
     %model of the filter
     
     pitch = atan(-acc_data(:,1)./(sqrt(acc_data(:,2).^2 + acc_data(:,3).^2))); 
     roll  = atan(acc_data(:,2)./acc_data(:,3));    
     
     pitch = [0,0,pitch'];
     roll =  [0,0,roll'];
     
     pitch_est_comp=[0.1,0.1];
     roll_est_comp=[0,0];
     for  i=1:(length(acc_data(:,1))-1)
     pitch_est_comp(i+2) = (1/delta_pit)*(Gamma_pit*pitch(i) +xi_pit*pitch(i+1) + eta_pit*pitch(i+2)+ sigma*(wy(i) - wy(i+2)) -alpha_pit*pitch_est_comp(i) - beta_pit*pitch_est_comp(i+1)); 
     roll_est_comp(i+2) = (1/delta_roll)*(Gamma_roll*roll(i) +xi_roll*roll(i+1) + eta_roll*roll(i+2)+ sigma*(wy(i) - wy(i+2)) -alpha_roll*roll_est_comp(i) - beta_roll*roll_est_comp(i+1)); 
     end
      \end{lstlisting}
      
   
   
      \subsection{Second Order Complementary Filter auto-tune code}\label{sec: auto-tune code}
      
      \begin{lstlisting} [caption=  MatLab Code Used to auto-tuning of the Complementary Filter, label=lst: auto_tuning of the Complementary Filter]
      function [Ki,Kp]=Tune_comp_filter(theta_input,acc_data,gyro_data,tau,ts,ki_fun_input,kp_fun_input,grid_size)  
      %
      %Inputs      :-
      %theta_input : is the actual angle which one wants the filter to output (input is in rads)
      %acc_data    : is an array of accelemeter data of all three axis (x,y,z)
      %gyro_data   : is an array of gyroscope data of all three axis (x,y,z) (inputs is in rads/second)
      %tau         : is the time delay of the complamentry filter
      %ts          : as this is a discrete filter one needs to give the time
      %              difference between each sample
      %ki_fun_input: is an array containing the min and max values of which the
      %              grid can use for Ki (sould be in the form [ki min, ki max])
      %kp_fun_input: is an array containing the min and max values of which the
      %              grid can use for Kp (sould be in the form [kp min, kp max])
      %grid_size   : the grid 
      %
      %Outputs     :- 
      %Ki          : returns the optiam Ki value for the given costs (RMSE and tau)
      %Kp          : returns the optiam Kp value for the given costs
      
      % the following defines the angle with repsct to the y-axis (pitch angle) 
      % as given by a three axis accelermeter and Euler (1-2-3) rotation from earth to frame  
      pitch = atan(-acc_data(:,1)./(sqrt(acc_data(:,2).^2 + acc_data(:,3).^2))); 
      
      %values as measured by the gyroscope (note the non-lineartise are not adjusted for)
      wx = [0;0;gyro_data(:,1)];
      wy = [0;0;gyro_data(:,2)];
      wz = [0;0;gyro_data(:,3)];
      
      pitch = [0,0,pitch'];
      
      % setting up the grid as required by the search
      kp_pit_array = linspace(kp_fun_input(1),kp_fun_input(2),grid_size);
      ki_pit_array = linspace(ki_fun_input(1),ki_fun_input(2),grid_size);
      previous_mean_error = 100000000;
      
      % grid search code
      num_times_changed = 1; % counter for the number of times kp and ki have changed
      for k = 1: (length(kp_pit_array)-1)
      
      kp_pit = kp_pit_array(k);
      
      for j = 1: (length(ki_pit_array)-1)
      ki_pit = ki_pit_array(j);
      
      theta = theta_input ;  
      
      % the following are filter weighting terms 
      Gamma_pit = ki_pit*ts^2 - 2*kp_pit*ts;
      alpha_pit = Gamma_pit + 4;
      xi_pit    = 2*ki_pit*ts^2;
      eta_pit   = ki_pit*ts^2 + 2*kp_pit*ts;
      beta_pit  = xi_pit - 8;
      delta_pit = eta_pit + 4;
      sigma = 2*ts;
      
      %reset angle at each loop (for robustnest to ensure no errors)
      angle_est_comp=[0,0];
      
      %calculation of the angle estimation (filtered anlge)
      for  i=1:(length(acc_data(:,1))-1)
      angle_est_comp(i+2) = (1/delta_pit)*(Gamma_pit*pitch(i) +xi_pit*pitch(i+1) + eta_pit*pitch(i+2)+ sigma*(wy(i) - wy(i+2)) -alpha_pit*angle_est_comp(i) - beta_pit*angle_est_comp(i+1));
      end
      
      %reduction of filtered angle so RMSE can be calculated
      angle_est_comp = angle_est_comp(1:(length(theta)));
      
      % the following removes any lag in the filter so one can plot the
      % results on top of each other so RMSE can be calculated 
      [C21,lag1] = xcorr(angle_est_comp,theta);
      [~,I1] = max(abs(C21));
      t31 = lag1(I1);
      if(t31 ==0)
      t31 = 1; 
      end    
      
      angle_est_comp = angle_est_comp(t31:end);
      theta = theta(1:(length(angle_est_comp)));
      
      mean_square_error   = sqrt(sum((theta - angle_est_comp').^2)/length(theta));
      
      tau_of_filter = 2/kp_pit; %time constant of the filter
      if (mean_square_error <previous_mean_error && tau_of_filter < tau)
      
      Kp = kp_pit;
      Ki = ki_pit;
      previous_mean_error = mean_square_error;
      %disp(['Number of times Kp and Ki have changed', num2str(num_times_changed)])
      num_times_changed = num_times_changed +1;
      end
      
      end
      %disp(['interation of outer loop:', num2str(k)])  
      end
      \end{lstlisting}
      
      
   
   \newpage
   \section{Kalman Filter (MatLab Code)}
   
   
      \subsection{Static Kalman Filter}\label{sec: static Kalman Filter}
      
      \begin{lstlisting} [caption=  MatLab Code Used to Implement a Static Kalman Filter, label=lst: Static Kalman Filter]
    function [K,F,H,filter_data]=kalman_filter_general(A_d,B_d,C_d,R,Q,u,z)
    
    %inputs  :-
    %A_d     : is the linear model of the state space system
    %B_d     : is the linear model of the input impacts states
    %C_d     : is the linear model of the measurements used by the filter
    %          relates to the systems states (commonly denoted as H for kalman filters)
    %R       : is the covariance matrix of the measurements (models the nosie of the sensors)
    %Q       : is the covariance matirx of the plantes      (modeld the noise in the states)
    %u       : is the plant input (can be in matrix form)
    %z       : is the avaible state measurements 
    %
    %outputs     :- 
    %K           : Kalman gain matrix
    %F           : F is the matrix that relates the estamate to the system model 
    %             (see gordon's control notes)
    %H           : H is the matrix that relates the estamate to the state measurement 
    %             (see gordon's control notes)
    %filter_data : is the filtered state or estimated state if one preferes
    
    xhat = [0;0];
    P = 100*eye(size(A_d));
    
    for i = 1:length(z)
    
    %Predict
    P_star = A_d*P*A_d' + Q;
    x_star = A_d*xhat + B_d*u(i);
    
    %correct
    K = P_star*C_d*inv(C_d*P_star*C_d' + R);
    xhat = x_star + K*(z(:,i) - C_d*x_star);
    P = (eye(size(A_d)) - K*C_d)*P_star;
    
    filter_data(:,i) = xhat;
    end
    
    F = (eye(size(A_d))-K*C_d)*A_d;
    H = (eye(size(A_d))-K*C_d)*B_d;
          \end{lstlisting}
          
   
   
   
   
   \subsection{Kinematic Kalman Filter}\label{sec: Kinematic Kalman Filter}
   
       \begin{lstlisting} [caption=  MatLab Code Used to Implement a Kinematic Kalman Filter, label=lst: Kinematic Kalman Filter]
     function [Angle_est] =Kinematic_kalman(acc_data,gyro_data,ts,R,Q,P) 
     % A adtive Kalman Filter whos A (transition) matrix changes with. This transition matrix is made from the gyroscope outputs
     % function [Angle_est] =Kinematic_kalman(acc_data,gyro_data,ts,R,Q,P)  
     %
     % Inputs:-
     %    acc_data    : A logged form of the accelerometer data (denoted $a_x$, $a_y$ and $a_z$ in this report, also this input has to be in an array formate)
     %    gyro_data   : A logged form of the gyroscope data (denoted $w_x$, $w_y$ and $w_z$ in this report, also this input has to be in an array formate)
     %    ts          : Sampleing of the controller used to collected the gyroscope and accelerometer data
     %    R           : Is the covariance matrix of the measurements (models the nosie of the sensors)
     %    Q           : Is the covariance matirx of the plantes      (models the noise in the states)
     %    P           : Initial guess of the covariance estimation error      
     %
     % Output:-  
     %    Angle_est      : Outputs an angle estimate for P,Q and R value
     
     for i = 1:length(ax)
     %Accelerometer measuremeterments        
     z = [ax(i);
     ay(i);
     az(i)];
     
     %Skew matrix, which uses the gyroscope to model the trasistion matrix
     A = [0          ,wz(i)   , -wy(i);
     -wz(i)     ,0       , wx(i);
     wy(i)      ,-wx(i)  ,0];
     
     %Discretize the continuous time A matrix so a digital estimation of the angle can be acquired       
     A_d=0;
     for q = 1:approx_to_matrix_exp
     A_d = A_d + (eye(size(A))*A^(q-1)*Ts^(q-1))/(factorial(q-1)); 
     end 
     
     %Predict
     P_star = A_d*P*A_d' + Q;
     x_star = A_d*xhat;
     
     %correct
     K = P_star*C_d*inv(C_d*P_star*C_d' + R);
     xhat = x_star + K*(z - C_d*x_star);
     P = (eye(size(A_d)) - K*C_d)*P_star;
     
     %Estimation of the angle is saved         
     Angle_est(i)      = atan(-xhat(1)/(sqrt(xhat(2)^2 + xhat(3)^2 )))*180/pi;
     end
    \end{lstlisting}
    
    
    
    
    
    
    
    