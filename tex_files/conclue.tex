\chapter{Conclusion}

A mathematical mode of the quad-rotor has be developed. Models for the accelerometer, gyroscope and magnetometer have been developed (as seen in  Section \ref{sec:imu section for relation of stuff})and proven to be correct by means of experiment. A \gls{gps} model has also been presented in Section \ref{sec: gps part} and can be used in future projects if a \gls{gps}-based device is used.


As the sensors used in the project are not ideal, the attitude of the quad-rotor had to be estimated by means of data fusion. These estimators have been tested in both simulation and on the quad-rotor.

The pitch and roll estimates were compared with actual values and the results showed that the filters worked as required an indeed produced an optimal response. Chapter \ref{chap: kalman implem} shows that the Kalman Filter is the optimal choice, but is more computationally intensive to run on some micro-controllers than a Complementary Filter. If the Kalman Filter is not an option, then a Complementary Filter similar to the ones presented in this report can be used for attitude estimate as they are less computational intensive.


The attitude estimators were designed and simulated using Simulink and MatLab. The results showed from these simulations demonstrated that these estimators could be implemented to control the quad-rotor. The Complementary Filter and Kalman Filer where both tested on the quad-rotor with a \gls{lqr} based control scheme and both were shown to give adequate  results.

Finally, the report gives a detailed description of how the filters compare to each other and how they can be implemented on a micro-controller. 


